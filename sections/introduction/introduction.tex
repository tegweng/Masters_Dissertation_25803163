\label{chapter:Introduction}

\section{Motivation}

Currently in oceanography the neutral surface is widely used as a basis for understanding mixing directions \citep{Tailleux2016}, \citep{McDougall1987}. In addition, in order to represent mixing in the ocean many models use a diffusion tensor based on the neutral direction \citep{McDougall2014}. This is based on arguments presented in \citet{McDougall1987} and \cite{JackettandMcDougall1997}, then extended in \citet{McDougall2014}. 



\section{Aims}
\label{section:intro_aims}

The aims of this project are to test elements of neutral surface theory using modern oceanographic measurements in the following ways:

\begin{itemize}
    \item Explore the mathematical arguments given in \citet{McDougall1987} in order to identify regimes in which we can categorically say whether the neutral surface is best aligned with the ``true" mixing direction  
    
    \item Reproduce original plots from \citet{McDougall1987} which show the spread of variables $\theta$ and $S$ on surfaces of interest and include the previously excluded $\sigma_4$ surface
    
    \item Extend these graphical arguments to the Gibraltar Straight in order to better align with the water masses known in this region (i.e. Mediterranean Intermediate Water entering the Atlantic)
    
    \item Investigate the energy cost on the surfaces of interest in the Atlantic region using two parcel theory to understand if the energy really is zero on the neutral surface and positive on potential density surfaces
\end{itemize}

\section{Outline of Dissertation}

In the first section of this dissertation we look into the background of the subject. This includes the definition of water masses and how it is currently understood interior mixing happens in the ocean. Then we introduce potential density and neutral surfaces as well as discussing the advantages and disadvantages of their use. 

In the next section 